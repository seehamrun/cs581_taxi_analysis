\begin{abstract}
Online advertising has become a driving force of today's economy, shaping the socio-economic and technological landscape with the provision of new online services and applications.
It continuously grows in an unprecedented rate, to the point that it has already surpassed other more traditional ways of reaching out to the people and 
promoting products and services.
However, this massive growth of online advertising has created a proportionate level of user tracking. 
Advertising companies track online users extensively to serve targeted advertisements.
The technology that drives online advertising, is called Real Time Bidding (RTB) with its most well-known protocol referred as ``Cookie Synchronization".
These protocols form the backbone of the web tracking and retargeting ecosystem, and are the important factors of todays synchronous Web.

Moreover, the rise of RTB has forced the Advertising and Analytics companies to collaborate more closely with one another, to exchange data about users to facilitate bidding in RTB auctions. 
Because of RTB, tracking data is not just observed by trackers embedded directly into web pages, but rather it is funneled through the advertising ecosystem through
complex networks of exchanges and auctions.
In either case, the implications for user privacy are severe: ad-companies are capable of tracking a user across all their digital space and screens, and use such information in a non-transparent fashion.

Measuring these flows of tracking information is challenging, and the works considered on this survey, have done the first significant steps towards a directions of understanding and measuring the online ad-ecosystem.
The first work is  by~\cite{bashir2016tracing}, where they propose a  methodology that can detect client- and server-side information flows between arbitrary tracking domains  using retargeted ads, and they  can successfully categorize four different kinds of information-sharing behaviors.
On a similar direction~\cite{papadopoulos2019cookie}, performed  a large scale study with real users, in order to understand and dissect the characteristics of the Cookie Synchronization protocol.
Finally, in a direction of modeling this complex eccosystem,~\cite{bashir2018diffusion} proposed a model  in the form of a graph called an Inclusion graph. Through measurements and simulations of the graph, they provide upper and lower estimates on the tracking information observed by advertising companies. 


\end{abstract}
